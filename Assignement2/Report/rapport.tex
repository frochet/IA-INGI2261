\documentclass[a4paper,10pt]{article}
\usepackage[utf8]{inputenc}
\usepackage[french] {babel}
\usepackage[T1]{fontenc}
\usepackage{lmodern}
\usepackage{graphicx}
\usepackage{graphics}
\usepackage{ulem}
\usepackage{amssymb}
\usepackage{url}
\usepackage[a4paper]{geometry}
\geometry{hscale=0.7,vscale=0.7,centering}
\usepackage{vmargin}
\usepackage{amsmath}
\usepackage{amssymb}
\usepackage{amsthm}
\usepackage{moreverb}
\usepackage{listings}
\newtheorem{theorem}{Théorème}[section]
\newtheorem{defi}{Définition}[section] 
\newtheorem{prop}{Propriété}[section] 
\usepackage{color}
\definecolor{gris}{rgb}{0.95,0.95,0.95}
\lstset{numbers=left, tabsize=4, backgroundcolor=\color{gris},
frame=single, breaklines=true,
keywordstyle=\color{black},
stringstyle=\ttfamily,
framexleftmargin=6mm, xleftmargin=6mm}
%opening
\title{LINGI 2261 : Artificial Intelligence \\
Assignement 2}
\author{Rochet Florentin - Debroux Léonard} 
\date{Année académique 2011-2012}

\begin{document}

	\begin{titlepage}
		\begin{center}
			{\huge LINGI2261: Artificial Intelligence}\\
			\vspace{0.4cm}
			
			{\Large {Professor : Yves Deville\\ \vspace{0.2cm} Teaching assistants : Cyrille Dejemeppe and Jean-Baptiste Mairy  }}\\
			\vspace{0.6cm}
			
			{\Large \textit{ Assignement2 : Solving Problems with Informed Search}}\\
			\vspace{1.2cm}

			\texttt{}\\
			\vspace{0.2cm}

			\includegraphics[height=10cm]{pageGarde.png}\\
			\vspace{0.1cm}
			{\Large \textbf{Universit\'e Catholique de Louvain}}
			\vspace{0.7cm}

			Groupe 37 \\
			\vspace{0.2cm}
			
			Florentin Rochet \\
			Léonard Debroux\\
			\vspace{0.2cm}
			2012-2013\\
		\end{center}
	\end{titlepage}

	\newpage
	
	\section{Search Algorithms}
		\subsection{Statments to be proven}
			\subsubsection{Question 1: Prove that depth-first is a special case of uniform-cost search}
				We have to find an example of a tree that have the same search-path for the depth-first and the uniform-cost. If a tree is build is such as the cost of the nodes increments by one following the path of an actual depth-search, the uniform-cost will follow that same path and thus perform a depth-first search. 
				
			\subsubsection{Question 2: Prove that breadth-first is a special case of uniform-cost search}	
				In the simple case where all the nodes have the exact same cost, the uniform-search will run through the tree by level exactly like a breadth-first search.
			\subsubsection{Question 3: Prove that greedy-best-first search is a special case of A*}
				The f(n) function of an A* search is g(n) + h(n), which is the sum of the exact cost from the beginning to a node n and the extimated cost provided by the heuristic. In a greedy-best first, g(n) is equal to zero and thus it is a special case of A* where we do not take into account the cost to get to the node n. 
			\subsubsection{Question 4: Prove that uniform-cost is a special case of A*}
				This is pretty much the same idea than Q3, only here, it is h(n) that is equal to zero which means that we do not use a heuristic. f(n) is equal only to g(n) that represent the cost from the beginning to the node n.
			
			\subsubsection{Question 5: Prove that if a heuristic is consistent, then it is admissible}
			
				\begin{proof}
					Let's procede by in a recursive way, using the consitency definition :						$$ h(n) \leq c(n,a,n') + h(n') $$
					We assume first that n' is the goal, and thus, h(g) equals 0 by definition. We get:
					$$ h(n) \leq c(n,a,g) $$
					Where n is the node strictly before reaching the goal. We see that h is admissible for this node. We can now start the recursion like so:
					$$ h(n) \leq c(n,a,n') + h(n') $$
					Knowing that h(n') is admissible, we can set an upper bound for h(n') that is the exact minimal cost to the goal and can write the above as:
					$$ h(n) \leq c(n,a,n') + C*(n', g) $$
					The right term can thus be written as:
					$$ h(n) \leq C*(n, g) $$
					Which is the definition of admissibility.        
				\end{proof}\\
				We have thus prove that a consistent heuristic is admissible.
				
		\subsection{A* versus uniform-cost search}
			\subsubsection{Question 1: Give a consistent heuristic for this problem. Prove that it is admissible}
				A consistent heuristic for this problem is to use h(n) = manhattan distance from n to the goal. We will prove that it is consistent and thanks to the result of question 1.5, we are sure that it will be admissible too.
				\begin{proof}
					To prove that this heuristic is consistent and thus, admissible, we consider that the consistency statement is false and get:
					$$ h(n) > c(n,a,n') + h(n') $$
					Where "a" is a move in any of the cardinal directions.
					By subtracting h(n') on both side we get:
					$$ h(n) - h(n') > c(n,a,n') $$
					The manhattan distance heuristic should have a cost of 1 per action and accordingly to the latter expression, it is at least of 2 because c(n,a,n') is equal to 1.\\
					This is impossible and thus, the statement of consistency must be true.\\
					Because the heuristic is consistent, it is admissible as proved earlier.   
				\end{proof}  				
				
			\subsubsection{Question 2: Show on the left maze the states (board positions) that are visited during an execution of A* graph search with a manhattan distance heuristic (ignoring walls). A state is visited when it is selected in the fringe and expanded. In the A* algorithm, we assume that when different states in the fringe have the smallest value, the algorithm chooses the state with the smallest coordinate (i, j) ((0, 0) being the bottom left position, i being the horizontal index and j the vertical one) using a lexicographical order}
				For the following search, when looking at the children of a node, we start looking to the north and continue clockwise.\\
			\begin{center}
				\begin{tabular}{|c|c|c|c|c|c|c|}
					\hline 
					  &    &    &    &    &    &   \\ 
					\hline 
					  &    &    &    &    &    &   \\ 
					\hline 
					  &    &    &    &    &    &   \\ 
					\hline 
					  &    &    &    &    &    &   \\ 
					\hline 
					9 & 10 & 11 & 12 &    &    &   \\ 
					\hline 
					7 & 8  & \# & 13 &    &    &   \\ 
					\hline 
					5 & 6  & \# & 14 & 17 & 19 &   \\ 
					\hline 
					2 & 4  & \# & 15 & 16 & 18 & 20 \\ 
					\hline 
					1 & 3  & \# &    &    &    &   \\ 
					\hline 
				\end{tabular}
			\end{center} 
				The path that is retained by the A* search using the manhattan distance heuristic is the following: \{1, 2, 4, 6, 8, 10, 11, 12, 13, 14, 15, 16, 18, 20\}. The numbers indicates in which order the nodes are seen.

			\subsubsection{Show on the right maze the board positions visited by a uniform-cost graph search. When several states have the smallest path cost, this uniform-cost search visits them in the same lexicographical order as A*}
			 \begin{center}
				\begin{tabular}{|c|c|c|c|c|c|c|}
					\hline 
					19 & 25 & 32 & 41 & 49 & 55 &    \\ 
					\hline 
					15 & 20 & 26 & 24 & 43 & 51 &    \\ 
					\hline 
					12 & 16 & 21 & 28 & 36 & 45 & 53 \\ 
					\hline 
					10 & 13 & 17 & 23 & 30 & 38 & 47 \\ 
					\hline 
					8  & 11 & 14 & 18 & 24 & 31 & 39 \\ 
					\hline 
					6  & 9  & \# & 22 & 29 & 37 & 46 \\ 
					\hline 
					4  & 7  & \# & 27 & 35 & 44 & 52 \\ 
					\hline 
					2  & 5  & \# & 33 & 42 & 50 & 56 \\ 
					\hline 
					1  & 3  & \# & 40 & 48 & 54 &    \\ 
					\hline 
				\end{tabular} 
			\end{center}
				The uniform cost search is less effective than the A* using a manhattan distance, though the path is also of length 13. That path follows the following set of nodes: \{1, 2, 4, 6, 8, 11, 14, 18, 22, 27, 33, 42, 50, 56\}
			
	\section{Sokoban planning problem}
	
		\subsection{Question 1: As illustrated on Figure 3 some situations cannot lead to a solution. Are there other similar situations? If yes, describe them.}
			
			There is another situation that cannot lead to a solution, that is when a block is against a wall and must go in a perpendicular direction w.r.t. that wall as shown on the representation below: \\
	\begin{tabular}{ccccc}
		\# & \# & \# & \# & \# \\ 
		\# &    &    &    & \# \\ 
		\# & .  &    & \$ & \# \\ 
		\# &    &    &    & \# \\ 
		\# & \# & \# & \# & \# \\ 
	\end{tabular}\\
	In this case, the box is stuck against the right wall and it will never be possible to push it to the left to reach the goal.\\
	In the case of a goal being against a wall, there is a dead state if two boxes a pushed against that wall as shown on the example below: \\
	\begin{tabular}{ccccccccc}
		\# & \# & \# & \# & \# & \# & \# & \# & \# \\ 
		\# &    & .  &    & \$ &    & \$ &    & \# \\ 
		\# &    &    &    &    &    &    &    & \# \\ 
		\# &    &    &    &    &    &    &    & \# \\ 
		\# & \# & \# & \# & \# & \# & \# & \# & \# \\ 
	\end{tabular}\\
	The box on the right will be able to go on the goal but the one on the right is in the same position as explain above, although there is a goal against the same wall as it.
		
		\subsection{Question 2: Why is it important to identify dead states in your successor function? How are you going to implement it?}
			
			It is important to identify the dead state because it avoids to generate and explore lots of useless states.\\
			 Let's assume that one box is on a dead state case, and two boxes can still be moved. Let's also say that there is about 20 reachable cases for the boxes left and for the character, the number of useless states would be around $ 20^{3} $ (not exactly) for only one box on one dead state. This is what shows the following example.\\
	\begin{tabular}{ccccccc}
		\# & \# & \# & \# & \# & \# & \# \\ 
		\# & \$ &    &    &    &    & \# \\ 
		\# &    & @  &    & \$ &    & \# \\ 
		\# &    &    & \$ &    &    & \# \\ 
		\# &    &    &    &    &    & \# \\
		\# & \# & \# & \# & \# & \# & \# \\ 
	\end{tabular}\\ 
			The upper left box can not be moved anymore.\\
			To avoid that kind of state, we created two type of dead state. Both are detected at the beginning of the execution but do not imply the same behaviour.
			
			\subsubsection{Static Dead States}
				If the avatar is trying to move a box on a static dead state, it is directly forbidden and the move is not allowed, it is especially the case in corner, as show on the example right above.
				Static dead cases exist also in other case but this will be covered in the next section.
				
			\subsubsection{Possible Dead States}
				Those are created when there is a goal next to a wall (see example 2 in 2.1). When there is no goal along a wall, then all the cases become static dead states, when there is one or several, they become possible dead states. This means that when there is as much boxes as goals along that wall there can not be one more. It avoids similar scenario's to example 2 in 2.1.\\
				It is implemented thanks to a list currentDeadStates proper to every state, it contains the coordinates of every possible dead state that has turned to dead state because of the boxes on it.\\
				The move actions update this list and allow moves for boxes that are on it.				
				
		\subsection{Question 3: Describe possible (non trivial) heuristic(s) to reach a goal state (with reference if any). Is(are) your heuristic(s) admissible and/or consistent?}
		
		\subsection{Question 4: Implement this problem. Extend the Problem class and implement the necessary methods and other class(es) if necessary. Your file must be named sokoban.py. You program must print to the standard output a solution to the sokoban instance given in argument satisfying the described format.}
			The resolution is launched using 
			\begin{quote}
				python3 sokoban.py sokoInst 
			\end{quote}
			There is no need to add the extensions.

		\subsection{Question 5: Experiment, compare and analyze informed (astar\_graph \_search) and uninformed (breadth\_first\_graph\_search) graph search of aima-python3 on the 15 instances of sokoban provided. Report in a table the time, the number of explored nodes and the number of steps to reach the solution. Are the number of explored nodes always smaller with astar\_graph\_search, why? When no solution can be found by a strategy in a reasonable time (say 5 min), explain the reason (time-out and/or swap of the memory).}
		
		
		\subsection{Question 6}
		
\end{document}