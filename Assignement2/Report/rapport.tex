\documentclass[a4paper,10pt]{article}
\usepackage[utf8]{inputenc}
\usepackage[french] {babel}
\usepackage[T1]{fontenc}
\usepackage{lmodern}
\usepackage{graphicx}
\usepackage{graphics}
\usepackage{ulem}
\usepackage{amssymb}
\usepackage{url}
\usepackage[a4paper]{geometry}
\geometry{hscale=0.7,vscale=0.7,centering}
\usepackage{vmargin}
\usepackage{amsmath}
\usepackage{amssymb}
\usepackage{amsthm}
\usepackage{moreverb}
\usepackage{listings}
\newtheorem{theorem}{Théorème}[section]
\newtheorem{defi}{Définition}[section] 
\newtheorem{prop}{Propriété}[section] 
\usepackage{color}
\definecolor{gris}{rgb}{0.95,0.95,0.95}
\lstset{numbers=left, tabsize=4, backgroundcolor=\color{gris},
frame=single, breaklines=true,
keywordstyle=\color{black},
stringstyle=\ttfamily,
framexleftmargin=6mm, xleftmargin=6mm}
%opening
\title{LINGI 2261 : Artificial Intelligence \\
Assignement 2}
\author{Rochet Florentin Debroux Léonard} 
\date{Année académique 2011-2012}

\begin{document}

	\begin{titlepage}
		\begin{center}
			{\huge LINGI2261: Artificial Intelligence}\\
			\vspace{0.4cm}
			
			{\Large {Professor : Yves Deville\\ \vspace{0.2cm} Teaching assistants : Cyrille Dejemeppe and Jean-Baptiste Mairy  }}\\
			\vspace{0.6cm}
			
			{\Large \textit{ Assignement2 : Solving Problems with Informed Search}}\\
			\vspace{1.2cm}

			\texttt{}\\
			\vspace{0.2cm}

			\includegraphics[height=10cm]{pageGarde.png}\\
			\vspace{0.1cm}
			{\Large \textbf{Universit\'e Catholique de Louvain}}
			\vspace{0.7cm}

			Groupe 37 \\
			\vspace{0.2cm}
			
			Florentin Rochet \\
			Léonard Debroux\\
			\vspace{0.2cm}
			2012-2013\\
		\end{center}
	\end{titlepage}

	\newpage
	
	\section{Search Algorithms}
		\subsection{Statments to be proven}
			\subsubsection{Question 1: Prove that depth-first is a special case of uniform-cost search}
				If the tree that is build is such as the cost of the nodes increments by one following the path of an actual depth-search, the uniform-cost will follow that same path and thus perform a depth-first search.
				
			\subsubsection{Question 2: Prove that breadth-first is a special case of uniform-cost search}	
				In the simple case where all the nodes have the exact same cost, the uniform-search will run through the tree by level exactly like a breadth-first search.
			\subsubsection{Question 3: Prove that greedy-best-first search is a special case of A*}
				The f(n) function of an A* search is g(n) + h(n), which is the sum of the exact cost from the beginning to a node n and the extimated cost provided by the heuristic. In a greedy-best first, g(n) is equal to zero and thus it is a special case of A* where we do not take into account the cost to get to the node n. 
			\subsubsection{Question 4: Prove that uniform-cost is a special case of A*}
				This is pretty much the same idea than Q3, only here, it is h(n) that is equal to zero which means that we do not use a heuristic. f(n) is equal only to g(n) that represent the cost from the beginning to the node n.
			
			\subsubsection{Question 5: Prove that if a heuristic is consistent, then it is admissible}
				
			
		\subsection{A* versus uniform-cost search}
		
	\section{Sokoban planning problem}
	
		\subsection{Question 1: As illustrated on Figure 3 some situations cannot lead to a solution. Are there other similar situations? If yes, describe them.}
			
			There is another situation that cannot lead to a solution, that is when a block is against a wall and must go in a perpendicular direction w.r.t. that wall as shown on the representation below: \\
	\begin{tabular}{ccccc}
		\# & \# & \# & \# & \# \\ 
		\# &    &    &    & \# \\ 
		\# & .  &    & \$ & \# \\ 
		\# &    &    &    & \# \\ 
		\# & \# & \# & \# & \# \\ 
	\end{tabular}\\
	In this cas, the box is stuck against the right wall and it will never be possible to push it to the left to reach the goal.
	In the of a goal being against a wall, there is a dead state if two boxes a pushed against that wall as shown on the exemple below: \\
	\begin{tabular}{ccccccccc}
		\# & \# & \# & \# & \# & \# & \# & \# & \# \\ 
		\# &    & .  &    & \$ &    & \$ &    & \# \\ 
		\# &    &    &    &    &    &    &    & \# \\ 
		\# &    &    &    &    &    &    &    & \# \\ 
		\# & \# & \# & \# & \# & \# & \# & \# & \# \\ 
	\end{tabular}\\
	The box on the right will be able to go on the goal but the one on the right is in the same position as explain above, although there is a goal against the same wall as it.
		
		\subsection{Question 2: Why is it important to identify dead states in your successor function? How are you going to implement it?}
		It's important because if the algorithm doesn't identify dead states, it cannot resolve the problem starting from some initial configuration. Once a block is in a dead state, it cannot move and the solution cannot be reached, this is a major problem. \\
		So we will handle dead state with some features. First of all, we will make a static Board which represent this type of case : 
		\begin{itemize}
			\item Wall 
			\item Goal
			\item Normal
			\item Static Dead state
		\end{itemize}
	So, one type of the case will be Static Dead state, which are typically corners on the map. It will be not allowed to push a box in this case.This static Board will be given to each object State that will be represent a particular state. This is this object State that will be yield by the method successor. And a State will be able to detect dynamic dead state around it. (Dead state which appear in some conditions)
			
		\subsection{Question 3}
		
		\subsection{Question 4}
		
		\subsection{Question 5}
		
		\subsection{Question 6}
		
\end{document}