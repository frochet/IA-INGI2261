\documentclass[a4paper,10pt]{article}
\usepackage[utf8]{inputenc}
\usepackage[french] {babel}
\usepackage[T1]{fontenc}
\usepackage{lmodern}
\usepackage{graphicx}
\usepackage{graphics}
\usepackage{ulem}
\usepackage{amssymb}
\usepackage{url}
\usepackage[a4paper]{geometry}
\geometry{hscale=0.7,vscale=0.7,centering}
\usepackage{vmargin}
\usepackage{amsmath}
\usepackage{amssymb}
\usepackage{amsthm}
\usepackage{moreverb}
\usepackage{listings}
\newtheorem{theorem}{Théorème}[section]
\newtheorem{defi}{Définition}[section] 
\newtheorem{prop}{Propriété}[section] 
\usepackage{color}
\definecolor{gris}{rgb}{0.95,0.95,0.95}
\lstset{numbers=left, tabsize=4, backgroundcolor=\color{gris},
frame=single, breaklines=true,
keywordstyle=\color{black},
stringstyle=\ttfamily,
framexleftmargin=6mm, xleftmargin=6mm}
%opening
\title{LINGI 2261 : Artificial Intelligence \\
Assignement 3 - Mid-Project}
\author{Rochet Florentin - Debroux Léonard} 
\date{Année académique 2011-2012}

\begin{document}

	\begin{titlepage}
		\begin{center}
			{\huge LINGI2261: Artificial Intelligence}\\
			\vspace{0.4cm}
			
			{\Large {Professor : Yves Deville\\ \vspace{0.2cm} Teaching assistants : Cyrille Dejemeppe and Jean-Baptiste Mairy  }}\\
			\vspace{0.6cm}
			
			{\Large \textit{ Assignement3 : Adversarial Search}}\\
			\vspace{1.2cm}

			\texttt{}\\
			\vspace{0.2cm}

			\includegraphics[height=10cm]{pageGarde.png}\\
			\vspace{0.1cm}
			{\Large \textbf{Universit\'e Catholique de Louvain}}
			\vspace{0.7cm}

			Groupe 37 \\
			\vspace{0.2cm}
			
			Florentin Rochet \\
			Léonard Debroux\\
			\vspace{0.2cm}
			2012-2013\\
		\end{center}
	\end{titlepage}

	\newpage
	

	\section{Scipion}
	
		\subsection{Draw the game tree for a depth of 2, i.e. one turn for each player.}
		\subsection{Evaluate the value of the leaves using the heuristic function (on the figure you
draw in 1, you don’t have to implement it!).}
		\subsection{Using the MiniMax algorithm, find the value of the other nodes.}
		\subsection{Circle the move the root player should do.}
		
	\section{Alpha-Beta search}	
		\subsection{Perform the MiniMax algorithm on the tree in Figure 4, i.e. put a value to each node. Circle the move the root player should do.}
		\subsection{Perform the Alpha-Beta algorithm on the tree. At each non terminal node, put the successive values of $\alpha$ and $\beta$. Cross out the arcs reaching non visited nodes. Assume a left-to-right node expansion.}
		\subsection{Do the same, assuming a right-to-left node expansion instead}
		\subsection{Can the nodes be ordered in such a way that Alpha-Beta pruning can cut off more branches (in a left-to-right node expansion)? If no, explain why; if yes,
give the new ordering and the resulting new pruning.}
		Yes it can. here the new ordering :
	\section{Sarena}
		\subsection{A basic Alpha-Beta agent}
		
		\subsection{Comparison of MiniMax and Alpha-Beta}
\end{document}